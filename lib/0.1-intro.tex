\setcounter{page}{1}

\addcontentsline{toc}{chapter}{INTRODUCCIÓN}
\chapter*{INTRODUCCIÓN}
\markboth{INTRODUCCIÓN}{}
\label{chap0:introducion}
\justify %%justificar los párrafos

\hspace*{1em} El continuo avance tecnológico en el ámbito de las comunicaciones ópticas ha posibilitado la transmisión eficiente de vastas cantidades de información a través de redes de fibra óptica. Aunque estas redes presentan numerosas ventajas, su despliegue no está exento de desafíos, especialmente derivados de fenómenos no lineales. La interacción entre la luz y el material de la fibra óptica da lugar a efectos no lineales, destacando su relevancia en sistemas de \acrfull{wdm}, donde múltiples canales de diferentes longitudes de onda coexisten con mínima separación. La alta intensidad óptica, consecuencia de la propagación simultánea de canales de alta potencia, puede desencadenar fenómenos críticos que impactan la eficacia de la comunicación óptica. Entre estos fenómenos no lineales, destaca la \acrfull{fwm}, un efecto óptico no lineal de tercer orden. La \acrshort{fwm} surge cuando dos o más señales ópticas con distintas frecuencias centrales se propagan en una misma fibra, generando armónicos que presentan frecuencias equivalentes a la suma o diferencia de las ondas originales.

....


\noindent{\textbf{Capítulo 1: Generalidades}}

En este capítulo se describen algunos aspectos generales sobre los sistemas de telecomunicaciones basados en fibra óptica. 
\vfill

\noindent{\textbf{Capítulo 2: Marco Metodológico}}

En este capítulo se definen la metodología y las herramientas de simulación 
\vfill

\noindent{\textbf{Capítulo 3: Diseño de la Arquitectura de Red}}

En este capítulo
\vfill

\noindent{\textbf{Capítulo 4: Diseño del Algoritmo}}

En este capítulo % \lipsum[2]
\vfill

\noindent{\textbf{Capítulo 5: Análisis del Desempeño de la Red Implementando el Algoritmo}}

En este capítulo se realiza el análisis del desempeño de los modelos 
\vfill

\noindent{\textbf{Capítulo 6: Conclusiones}}

En este capítulo se presentan las conclusiones, recomendaciones y trabajos futuros realizados 
\vfill


\noindent{\textbf{Palabras clave:}}

Mezcla de Cuadro Ondas (FWM), Multiplexación por División de Longitud de Onda, Monitoreo de Desempeño Óptico (OPM).
