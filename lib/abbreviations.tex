%%------------- ACROYNMS ---------------
%--An acronym is a type of abbreviation formed from the initial letters of a phrase and pronounced as a word.
%--For example NASA (National Aeronautics and Space Administration)
%--all acronyms are abbreviations, but not all abbreviations are acronyms.
\newacronym{gcd}{GCD}{Greatest Common Divisor}
\newacronym{lcm}{LCM}{Least Common Multiple}
\newacronym{pmi}{PMI}{Project Management Institute}
\newacronym{pmbok}{PMBOK}{Project Management Body of Knowledge}
\newacronym{tic}{TIC}{Tecnologías de la Información y de las Comunicaciones}
\newacronym{fwm}{FWM}{Mezcla de Cuatro Ondas \textit{(Four Wave Mixing)}}
\newacronym{MLR-PON}{MLR-PON}{Redes de Multiplexación por Longitud de Onda Pasiva Reconfigurable (\textit{Multi-Level Resilience Passive Optical Network})}
\newacronym{USCA}{USCA}{Supresión de Cruce de Señales Acústicas}
\newacronym{wdm}{WDM}{Multiplexación por División de Longitud de Onda}
\newacronym{pon}{PON}{Red Óptica Pasiva}


%%---- ABBREVIATIONS ----
%--An abbreviation is a shortened form of a word or phrase.
%--For example, the word abbreviation can itself be represented by the abbreviation abbr., abbrv., or abbrev.
\newglossaryentry{latex}
{
    name=latex,
    description={Is a markup language specially suited
            for scientific documents}
}
\newglossaryentry{maths}
{
    name=mathematics,
    description={Mathematics is what mathematicians do}
}
\newglossaryentry{FWM}
{
    name=mezcla de cuatro ondas,
    description={Mezcla de Cuatro Ondas}
}

%%---- HOW TO USE ----
%--\gls{gcd} -> GCD
%--\gls{latex} -> latex - minúscula

%--\Gls{gcd} -> Greatest Common Divisor (GCD) - same as \acrfull{gcd}
%--\Gls{latex} -> Latex - mayúscula

%--\glspl{gcd} -> Greatest Common Divisors (GCDs)
%--\glspl{latex} -> Latexes

%--\Glspl{gcd} -> Greatest Common Divisors (GCDs)
%--\Glspl{latex} -> Latexes

%--\acrshort{gcd} -> GCD
%--\acrlong{gcd} -> Greatest Common Divisor
%--\acrfull{gcd} -> Greatest Common Divisor (GCD)
