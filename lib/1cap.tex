\chapter[GENERALIDADES]{\Large GENERALIDADES}
\label{chap1:generalidades}
\justify %%justificar los párrafos

Una vez definido el motivo y los objetivos del trabajo y antes de dar un enfoque
directamente sobre el problema, es necesario saber la situación actual del mismo.
Comprender el objetivo, las tecnologías y arquitecturas utilizadas, es esencial para conocer el contexto del problema. En este capítulo hablaremos sobre los conceptos principales que se van a tratar a lo largo del proyecto.



\section{Four Wave Mixing (FWM)}
La Mezcla de Cuatro Ondas (FWM, por sus siglas en inglés \textit{Four Wave Mixing}) es un fenómeno no lineal que se manifiesta en sistemas de fibra óptica. Este efecto surge cuando dos o más señales ópticas, con frecuencias centrales diferentes, coexisten y se propagan a lo largo de una misma fibra.

En condiciones normales, estas señales ópticas deberían viajar independientemente sin interactuar significativamente entre sí. Sin embargo, debido a las no linealidades inherentes a la fibra óptica, como la no linealidad Kerr, las señales pueden influenciarse mutuamente.

El proceso de FWM implica la generación de nuevas componentes de frecuencia en la señal óptica original. Específicamente, se generan sumas y diferencias de las frecuencias originales, dando lugar a componentes espectrales adicionales. Este fenómeno puede introducir interferencias indeseadas entre los canales de comunicación en sistemas de multiplexación por división de longitud de onda (WDM), afectando negativamente la calidad de la señal y, en última instancia, la integridad de la transmisión de datos.

La FWM se convierte en un desafío significativo, especialmente en entornos donde se utilizan múltiples canales con frecuencias cercanas, como en sistemas WDM con canales igualmente espaciados. La supresión efectiva de la FWM es esencial para garantizar un rendimiento óptimo en las redes ópticas y maximizar la capacidad del sistema.

En el contexto de este trabajo, la propuesta de un mecanismo dinámico para contrarrestar la FWM en una red MLR-PON busca abordar este desafío, utilizando estrategias como la asignación de canales desigualmente espaciados y algoritmos optimizados para mitigar los efectos perjudiciales de la FWM.





